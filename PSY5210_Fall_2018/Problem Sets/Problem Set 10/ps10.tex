\documentclass[]{article}
\usepackage{lmodern}
\usepackage{amssymb,amsmath}
\usepackage{ifxetex,ifluatex}
\usepackage{fixltx2e} % provides \textsubscript
\ifnum 0\ifxetex 1\fi\ifluatex 1\fi=0 % if pdftex
  \usepackage[T1]{fontenc}
  \usepackage[utf8]{inputenc}
\else % if luatex or xelatex
  \ifxetex
    \usepackage{mathspec}
  \else
    \usepackage{fontspec}
  \fi
  \defaultfontfeatures{Ligatures=TeX,Scale=MatchLowercase}
\fi
% use upquote if available, for straight quotes in verbatim environments
\IfFileExists{upquote.sty}{\usepackage{upquote}}{}
% use microtype if available
\IfFileExists{microtype.sty}{%
\usepackage{microtype}
\UseMicrotypeSet[protrusion]{basicmath} % disable protrusion for tt fonts
}{}
\usepackage[margin=1in]{geometry}
\usepackage{hyperref}
\hypersetup{unicode=true,
            pdftitle={Problem Set 10},
            pdfauthor={Shane T. Mueller shanem@mtu.edu},
            pdfborder={0 0 0},
            breaklinks=true}
\urlstyle{same}  % don't use monospace font for urls
\usepackage{color}
\usepackage{fancyvrb}
\newcommand{\VerbBar}{|}
\newcommand{\VERB}{\Verb[commandchars=\\\{\}]}
\DefineVerbatimEnvironment{Highlighting}{Verbatim}{commandchars=\\\{\}}
% Add ',fontsize=\small' for more characters per line
\usepackage{framed}
\definecolor{shadecolor}{RGB}{248,248,248}
\newenvironment{Shaded}{\begin{snugshade}}{\end{snugshade}}
\newcommand{\KeywordTok}[1]{\textcolor[rgb]{0.13,0.29,0.53}{\textbf{#1}}}
\newcommand{\DataTypeTok}[1]{\textcolor[rgb]{0.13,0.29,0.53}{#1}}
\newcommand{\DecValTok}[1]{\textcolor[rgb]{0.00,0.00,0.81}{#1}}
\newcommand{\BaseNTok}[1]{\textcolor[rgb]{0.00,0.00,0.81}{#1}}
\newcommand{\FloatTok}[1]{\textcolor[rgb]{0.00,0.00,0.81}{#1}}
\newcommand{\ConstantTok}[1]{\textcolor[rgb]{0.00,0.00,0.00}{#1}}
\newcommand{\CharTok}[1]{\textcolor[rgb]{0.31,0.60,0.02}{#1}}
\newcommand{\SpecialCharTok}[1]{\textcolor[rgb]{0.00,0.00,0.00}{#1}}
\newcommand{\StringTok}[1]{\textcolor[rgb]{0.31,0.60,0.02}{#1}}
\newcommand{\VerbatimStringTok}[1]{\textcolor[rgb]{0.31,0.60,0.02}{#1}}
\newcommand{\SpecialStringTok}[1]{\textcolor[rgb]{0.31,0.60,0.02}{#1}}
\newcommand{\ImportTok}[1]{#1}
\newcommand{\CommentTok}[1]{\textcolor[rgb]{0.56,0.35,0.01}{\textit{#1}}}
\newcommand{\DocumentationTok}[1]{\textcolor[rgb]{0.56,0.35,0.01}{\textbf{\textit{#1}}}}
\newcommand{\AnnotationTok}[1]{\textcolor[rgb]{0.56,0.35,0.01}{\textbf{\textit{#1}}}}
\newcommand{\CommentVarTok}[1]{\textcolor[rgb]{0.56,0.35,0.01}{\textbf{\textit{#1}}}}
\newcommand{\OtherTok}[1]{\textcolor[rgb]{0.56,0.35,0.01}{#1}}
\newcommand{\FunctionTok}[1]{\textcolor[rgb]{0.00,0.00,0.00}{#1}}
\newcommand{\VariableTok}[1]{\textcolor[rgb]{0.00,0.00,0.00}{#1}}
\newcommand{\ControlFlowTok}[1]{\textcolor[rgb]{0.13,0.29,0.53}{\textbf{#1}}}
\newcommand{\OperatorTok}[1]{\textcolor[rgb]{0.81,0.36,0.00}{\textbf{#1}}}
\newcommand{\BuiltInTok}[1]{#1}
\newcommand{\ExtensionTok}[1]{#1}
\newcommand{\PreprocessorTok}[1]{\textcolor[rgb]{0.56,0.35,0.01}{\textit{#1}}}
\newcommand{\AttributeTok}[1]{\textcolor[rgb]{0.77,0.63,0.00}{#1}}
\newcommand{\RegionMarkerTok}[1]{#1}
\newcommand{\InformationTok}[1]{\textcolor[rgb]{0.56,0.35,0.01}{\textbf{\textit{#1}}}}
\newcommand{\WarningTok}[1]{\textcolor[rgb]{0.56,0.35,0.01}{\textbf{\textit{#1}}}}
\newcommand{\AlertTok}[1]{\textcolor[rgb]{0.94,0.16,0.16}{#1}}
\newcommand{\ErrorTok}[1]{\textcolor[rgb]{0.64,0.00,0.00}{\textbf{#1}}}
\newcommand{\NormalTok}[1]{#1}
\usepackage{graphicx,grffile}
\makeatletter
\def\maxwidth{\ifdim\Gin@nat@width>\linewidth\linewidth\else\Gin@nat@width\fi}
\def\maxheight{\ifdim\Gin@nat@height>\textheight\textheight\else\Gin@nat@height\fi}
\makeatother
% Scale images if necessary, so that they will not overflow the page
% margins by default, and it is still possible to overwrite the defaults
% using explicit options in \includegraphics[width, height, ...]{}
\setkeys{Gin}{width=\maxwidth,height=\maxheight,keepaspectratio}
\IfFileExists{parskip.sty}{%
\usepackage{parskip}
}{% else
\setlength{\parindent}{0pt}
\setlength{\parskip}{6pt plus 2pt minus 1pt}
}
\setlength{\emergencystretch}{3em}  % prevent overfull lines
\providecommand{\tightlist}{%
  \setlength{\itemsep}{0pt}\setlength{\parskip}{0pt}}
\setcounter{secnumdepth}{0}
% Redefines (sub)paragraphs to behave more like sections
\ifx\paragraph\undefined\else
\let\oldparagraph\paragraph
\renewcommand{\paragraph}[1]{\oldparagraph{#1}\mbox{}}
\fi
\ifx\subparagraph\undefined\else
\let\oldsubparagraph\subparagraph
\renewcommand{\subparagraph}[1]{\oldsubparagraph{#1}\mbox{}}
\fi

%%% Use protect on footnotes to avoid problems with footnotes in titles
\let\rmarkdownfootnote\footnote%
\def\footnote{\protect\rmarkdownfootnote}

%%% Change title format to be more compact
\usepackage{titling}

% Create subtitle command for use in maketitle
\newcommand{\subtitle}[1]{
  \posttitle{
    \begin{center}\large#1\end{center}
    }
}

\setlength{\droptitle}{-2em}

  \title{Problem Set 10}
    \pretitle{\vspace{\droptitle}\centering\huge}
  \posttitle{\par}
    \author{Shane T. Mueller
\href{mailto:shanem@mtu.edu}{\nolinkurl{shanem@mtu.edu}}}
    \preauthor{\centering\large\emph}
  \postauthor{\par}
      \predate{\centering\large\emph}
  \postdate{\par}
    \date{November 27, 2018}


\begin{document}
\maketitle

\texttt{This\ problem\ set\ covers\ simple\ categorical\ predictors,\ the\ link\ between\ regression\ and\ ANOVA,\ and\ post-hoc\ tests,\ and\ multi-way\ ANOVA.}

\section{Categorical Predictors}\label{categorical-predictors}

On each of day of one week, we sampled 100 random company stocks and
examined their trading price. Each day a different set of stocks was
sampled at random from the NYSE and NASDAQ published prices.

\begin{Shaded}
\begin{Highlighting}[]
\KeywordTok{library}\NormalTok{(ggplot2)}
\NormalTok{data <-}\StringTok{ }\KeywordTok{read.csv}\NormalTok{(}\StringTok{"ps10data.csv"}\NormalTok{)}

\KeywordTok{head}\NormalTok{(data)}
\end{Highlighting}
\end{Shaded}

\begin{verbatim}
##   day     sector stockprice
## 1  Mo automotive      55.11
## 2  Mo automotive      85.10
## 3  Mo     health      99.67
## 4  Mo automotive      19.79
## 5  Mo automotive      69.68
## 6  Mo automotive      61.97
\end{verbatim}

This is stored in a matrix. For a regression or ANOVA, we really need
each one

\begin{Shaded}
\begin{Highlighting}[]
\KeywordTok{ggplot}\NormalTok{(data,}\KeywordTok{aes}\NormalTok{(}\DataTypeTok{x=}\NormalTok{day,}\DataTypeTok{y=}\NormalTok{stockprice)) }\OperatorTok{+}\StringTok{ }\KeywordTok{geom_point}\NormalTok{(}\KeywordTok{aes}\NormalTok{(}\DataTypeTok{color=}\NormalTok{sector)) }\OperatorTok{+}\KeywordTok{theme_minimal}\NormalTok{()}
\end{Highlighting}
\end{Shaded}

\includegraphics{ps10_files/figure-latex/unnamed-chunk-2-1.pdf} For this
problem, we want to determine, using a number of methods, which days
differed from which other days. In each case, run the test, and answer
the question in 1-2 sentences describing what you found. Use a p=.05 as
a criterion for determining whether an effect isstatistically
significant.

\subsection{1. First use a contrast that will compare each day to
Monday, and report which of the days had prices significantly higher
than monday (report the test obtained directly from the coefficients of
lm by doing summary() on the results of
lm()).}\label{first-use-a-contrast-that-will-compare-each-day-to-monday-and-report-which-of-the-days-had-prices-significantly-higher-than-monday-report-the-test-obtained-directly-from-the-coefficients-of-lm-by-doing-summary-on-the-results-of-lm.}

\begin{Shaded}
\begin{Highlighting}[]
\CommentTok{#Question1}
\NormalTok{day.}\DecValTok{0}\NormalTok{ <-}\StringTok{ }\KeywordTok{c}\NormalTok{(}\StringTok{"Mo"}\NormalTok{,}\StringTok{"Tu"}\NormalTok{,}\StringTok{"We"}\NormalTok{,}\StringTok{"Th"}\NormalTok{,}\StringTok{"Fr"}\NormalTok{,}\StringTok{"Sa"}\NormalTok{,}\StringTok{"Su"}\NormalTok{)}
\NormalTok{data}\OperatorTok{$}\NormalTok{day<-}\StringTok{ }\KeywordTok{factor}\NormalTok{(data}\OperatorTok{$}\NormalTok{day,}\DataTypeTok{levels=}\NormalTok{day.}\DecValTok{0}\NormalTok{) }\CommentTok{#add the level to months variable}
\KeywordTok{aggregate}\NormalTok{(data}\OperatorTok{$}\NormalTok{stockprice,}\KeywordTok{list}\NormalTok{(data}\OperatorTok{$}\NormalTok{day),mean)}
\end{Highlighting}
\end{Shaded}

\begin{verbatim}
##   Group.1       x
## 1      Mo 60.5618
## 2      Tu 59.5182
## 3      We 60.2386
## 4      Th 80.7623
## 5      Fr 81.5663
## 6      Sa 78.4041
## 7      Su 83.7771
\end{verbatim}

\begin{Shaded}
\begin{Highlighting}[]
\NormalTok{model1 <-}\StringTok{ }\KeywordTok{lm}\NormalTok{(stockprice}\OperatorTok{~}\NormalTok{day, }\DataTypeTok{data=}\NormalTok{data)}
\NormalTok{model1}
\end{Highlighting}
\end{Shaded}

\begin{verbatim}
## 
## Call:
## lm(formula = stockprice ~ day, data = data)
## 
## Coefficients:
## (Intercept)        dayTu        dayWe        dayTh        dayFr  
##     60.5618      -1.0436      -0.3232      20.2005      21.0045  
##       daySa        daySu  
##     17.8423      23.2153
\end{verbatim}

\begin{Shaded}
\begin{Highlighting}[]
\KeywordTok{summary}\NormalTok{(model1)}
\end{Highlighting}
\end{Shaded}

\begin{verbatim}
## 
## Call:
## lm(formula = stockprice ~ day, data = data)
## 
## Residuals:
##     Min      1Q  Median      3Q     Max 
## -90.497 -33.876  -0.053  36.118 103.973 
## 
## Coefficients:
##             Estimate Std. Error t value Pr(>|t|)    
## (Intercept)  60.5618     4.2292  14.320  < 2e-16 ***
## dayTu        -1.0436     5.9809  -0.174 0.861533    
## dayWe        -0.3232     5.9809  -0.054 0.956920    
## dayTh        20.2005     5.9809   3.377 0.000772 ***
## dayFr        21.0045     5.9809   3.512 0.000474 ***
## daySa        17.8423     5.9809   2.983 0.002953 ** 
## daySu        23.2153     5.9809   3.882 0.000114 ***
## ---
## Signif. codes:  0 '***' 0.001 '**' 0.01 '*' 0.05 '.' 0.1 ' ' 1
## 
## Residual standard error: 42.29 on 693 degrees of freedom
## Multiple R-squared:  0.05869,    Adjusted R-squared:  0.05054 
## F-statistic: 7.202 on 6 and 693 DF,  p-value: 1.808e-07
\end{verbatim}

\subsection{2. Then, use successive difference coding of the day
variable to determine which days of the week differed significantly from
the previous
day.}\label{then-use-successive-difference-coding-of-the-day-variable-to-determine-which-days-of-the-week-differed-significantly-from-the-previous-day.}

\begin{Shaded}
\begin{Highlighting}[]
\CommentTok{#Question2}
\KeywordTok{library}\NormalTok{(MASS)}
\KeywordTok{contrasts}\NormalTok{(data}\OperatorTok{$}\NormalTok{day)<-}\KeywordTok{contr.sdif}\NormalTok{(}\KeywordTok{levels}\NormalTok{(data}\OperatorTok{$}\NormalTok{day))}
\NormalTok{model2 <-}\StringTok{ }\KeywordTok{lm}\NormalTok{(stockprice}\OperatorTok{~}\NormalTok{day, }\DataTypeTok{data=}\NormalTok{data)}
\NormalTok{model2}
\end{Highlighting}
\end{Shaded}

\begin{verbatim}
## 
## Call:
## lm(formula = stockprice ~ day, data = data)
## 
## Coefficients:
## (Intercept)     dayTu-Mo     dayWe-Tu     dayTh-We     dayFr-Th  
##     72.1183      -1.0436       0.7204      20.5237       0.8040  
##    daySa-Fr     daySu-Sa  
##     -3.1622       5.3730
\end{verbatim}

\begin{Shaded}
\begin{Highlighting}[]
\KeywordTok{summary}\NormalTok{(model2)}
\end{Highlighting}
\end{Shaded}

\begin{verbatim}
## 
## Call:
## lm(formula = stockprice ~ day, data = data)
## 
## Residuals:
##     Min      1Q  Median      3Q     Max 
## -90.497 -33.876  -0.053  36.118 103.973 
## 
## Coefficients:
##             Estimate Std. Error t value Pr(>|t|)    
## (Intercept)  72.1183     1.5985  45.117  < 2e-16 ***
## dayTu-Mo     -1.0436     5.9809  -0.174 0.861533    
## dayWe-Tu      0.7204     5.9809   0.120 0.904162    
## dayTh-We     20.5237     5.9809   3.432 0.000636 ***
## dayFr-Th      0.8040     5.9809   0.134 0.893104    
## daySa-Fr     -3.1622     5.9809  -0.529 0.597174    
## daySu-Sa      5.3730     5.9809   0.898 0.369309    
## ---
## Signif. codes:  0 '***' 0.001 '**' 0.01 '*' 0.05 '.' 0.1 ' ' 1
## 
## Residual standard error: 42.29 on 693 degrees of freedom
## Multiple R-squared:  0.05869,    Adjusted R-squared:  0.05054 
## F-statistic: 7.202 on 6 and 693 DF,  p-value: 1.808e-07
\end{verbatim}

\subsection{3. Use pairwise.t.test function to compute all pairwise
t-tests and the holm correction between days of the week. Describe
concisely which days differed from which other
days.}\label{use-pairwise.t.test-function-to-compute-all-pairwise-t-tests-and-the-holm-correction-between-days-of-the-week.-describe-concisely-which-days-differed-from-which-other-days.}

\subsection{4. Use an aov() model to predict stock price by day, and
then compute Tukey HSD test on all pairwise comparisons using the Tukey
test. Do the result differ from part
3?}\label{use-an-aov-model-to-predict-stock-price-by-day-and-then-compute-tukey-hsd-test-on-all-pairwise-comparisons-using-the-tukey-test.-do-the-result-differ-from-part-3}

\subsection{5. Compute a kruskall-wallis test to see if the
non-parametric test shows stock price depended on
day-of-week.}\label{compute-a-kruskall-wallis-test-to-see-if-the-non-parametric-test-shows-stock-price-depended-on-day-of-week.}

\subsection{6. Compute a one-way BayesFactor ANOVA and report the Bayes
factor score determining if day-of-week impacted stock
price.}\label{compute-a-one-way-bayesfactor-anova-and-report-the-bayes-factor-score-determining-if-day-of-week-impacted-stock-price.}

\section{2. Multi-way ANOVA and
regression.}\label{multi-way-anova-and-regression.}

The stocks were sampled from two different sectors (health and
automotive). Was there a difference in outcome based on sector? What
about when day day-of-week is considered. Report a standard (Type-I)
ANOVA F-test for:

\subsection{1. the effect of sector on its own (a one-way test),
and}\label{the-effect-of-sector-on-its-own-a-one-way-test-and}

\subsection{\texorpdfstring{2. whether sector has an effect \emph{after}
day-of-week is considered:
lm(stockprice\textasciitilde{}day+sector)}{2. whether sector has an effect after day-of-week is considered: lm(stockprice\textasciitilde{}day+sector)}}\label{whether-sector-has-an-effect-after-day-of-week-is-considered-lmstockpricedaysector}

\subsection{3. whether the results differ if sector is included in the
model first
(lm(stockprice\textasciitilde{}sector+day))}\label{whether-the-results-differ-if-sector-is-included-in-the-model-first-lmstockpricesectorday}

Then compare results of the three tests, including the sum-squared
deviations and the results of the F test. Are the results of the tests
identical or do they differ? Why? Pick which one you would prefer to use
to test the effect, and describe why you feel it is better than the
others.


\end{document}
